\section{Owiki Gries Logic}
Give a (non-trivial) specification for the following program and prove it using Owicki-Griel logic
\begin{equation*}
	\{X=A\land Y=B\}x:=X;Y:=1 ||y :=Y; X:=x\ldots
\end{equation*}
\begin{proof}
We will prove the following specification for the program:
\begin{equation*}
	\hoare{X=A\land Y=B}{x:=X;Y:=1 || y:=Y;X:=x}{x=A\lor y=B}
\end{equation*}
First, add two auxiliary variables $done_0$ and $done_1$ that record the state of execution of first and second statements accordingly. Now, rewrite the program as:
\begin{equation*}
done_0=0;done_1=0;(x:=X;(Y,done_0):=(1,1) || y:=Y;(X,done_1):=(x,1))
\end{equation*}
and let
\begin{gather*}
S_0=x:=X;(Y,done_0):=(1,1)\\
S_1=y:=Y;(X,done_1):=(x,1)\\
S_{00}=x:=X\\
S_{01}=(Y,done_0):=(1,1)\\
S_{10}=y:=Y\\
S_{11}=(X,done_1):=(x,1)
\end{gather*}
Now we can define the pre conditions of each sub statement:
\begin{gather*}
P_{00}=(done_0=0\land (done_1=0\rightarrow X=A)\land (done_1=1\rightarrow (x=A\lor y=B))) \\
P_{01}=(done_0=0\land (done_1=0\rightarrow x=A)\land (done_1=1\rightarrow (x=A\lor y=B))) \\
P_{10}=(done_1=0\land (done_0=0\rightarrow Y=B)\land (done_0=1\rightarrow (x=A\lor y=B))) \\
P_{11}=(done_1=0\land (done_0=0\rightarrow y=B)\land (done_0=1\rightarrow (x=A\lor y=B)))
\end{gather*}
Also, define the post conditions:
\begin{gather*}
Q_{0}=(done_0=1\land (done_1=0\rightarrow x=A)\land (done_1=1\rightarrow (x=A\lor y=B))) \\
Q_{1}=(done_1=1\land (done_0=0\rightarrow y=B)\land (done_0=1\rightarrow (x=A\lor y=B)))
\end{gather*}
$P_{00}$ and $P_{10}$ describe the state of events before the assignment to $x$ or $y$ took place, and $P_{10}$ and $P_{11}$ after, accordingly. After the counterpart statement is done, we can have both $x$ and $y$ updated, or only one of them.\\ \\
Now we wish to show that $S_0$ and $S_1$ are interference free. To do this we need to prove the following to show that $S_0$ is interference free:
\begin{gather}
\hoare{P_{00}\land Q_1}{S_0}{Q_1}\\
\hoare{P_{00}\land P_{10}}{S_0}{P_{10}}\\
\hoare{P_{00}\land P_{11}}{S_0}{P_{11}}
\end{gather}
Theorem (1) is the post condition test and theorems (2) and (3) are the pre condition tests that check each "sub" pre-condition in $S_1$. And we need to prove the following to show that $S_1$ is interference free:
\begin{align}
\hoare{P_{10}\land Q_0}{S_1}{Q_0}\\
\hoare{P_{10}\land P_{00}}{S_1}{P_{00}}\\
\hoare{P_{10}\land P_{01}}{S_1}{P_{01}}
\end{align}
We are going to use the following identities:
\begin{gather*}
Q_0\land Q_1=(x=A\lor y=B)\land done_0=1\land done_1=1 \\
P_{00}\land Q_1=(x=A\lor y=B)\land done_0=0\land done_1=1 \\
P_{10}\land Q_0=(x=A\lor y=B)\land done_0=1\land done_1=0 \\
P_{00}\land P_{10}=X=A\land Y=B\land done_0=0\land done_1=0 \\
P_{00}\land P_{11}=X=A\land y=B\land done_0=0\land done_1=0 \\
P_{10}\land P_{01}=x=A\land Y=B\land done_0=0\land done_1=0
\end{gather*}
\begin{itemize}
\item For (1), we have:
\begin{gather*}
\{(x=A\lor y=B)\land done_0=0\land done_1=1\}\\
x:=X;(Y,done_0):=(1,1)\\
\{done_1=1\land (done_0=0\rightarrow y=B)\land (done_0=1\rightarrow (x=A\lor y=B))\}
\end{gather*}
We see that $done_0=1$ and $y=B$ after the execution and obviously $x=A\lor y=B$.
\item For (2), we have
\begin{gather*}
\{X=A\land Y=B\land done_0=0\land done_1=0\}\\
x:=X;(Y,done_0):=(1,1)\\
\{done_1=0\land (done_0=0\rightarrow Y=B)\land (done_0=1\rightarrow (x=A\lor y=B))\}
\end{gather*}
Using the assignment axiom ($\hoare{X=A}{x:=X}{x=A}$ if we omit the auxiliary variables for clarity sake), we get that $x=A$ holds after the execution. Also $done_0=1$, and obviously $x=A\lor y=B$.
\item For (3), we have
\begin{gather*}
\{X=A\land y=B\land done_0=0\land done_1=0\}\\
x:=X;(Y,done_0):=(1,1)\\
\{done_1=0\land (done_0=0\rightarrow y=B)\land (done_0=1\rightarrow (x=A\lor y=B))\}
\end{gather*}
Exactly as (2). $done_0=1$ after the execution, $x=A$ holds and obviously $x=A\lor y=B$.
\end{itemize}
The idea behind (4), (5) and (6) is symmetrical to (1), (2) and (3), so the proof is skipped.\\ \\
Now we can show the inference tree for the program in two parts. First of all, we have:
\begin{equation}
\inferrule
{
	\inferrule
	{
		\hoare{P_{00}}{S_{00}}{P_{01}} \\ \hoare{P_{01}}{S_{01}}{Q_0}
	}
	{
		\hoare{P_{00}}{S_0}{Q_0}
	}
	\\
	\inferrule
	{
		\hoare{P_{10}}{S_{10}}{P_{11}} \\ \hoare{P_{11}}{S_{11}}{Q_1}
	}
	{
		\hoare{P_{10}}{S_1}{Q_1}
	}
}
{
	\hoare{P_{00}\land P_{10}}{S_0||S_1}{Q_0\land Q_1}
}
\end{equation}
First line contains Hoare logic axioms of assignment. Transition from first to second line is rule of composition. Transition from second to third line is Owicki-Gries rule for interference free statements. We will use the intermediate result of (7) in the next part:
\begin{equation*}
\inferrule
{
	\inferrule
	{
		\hoare{P_{00}\land P_{10}}{done_0=0;done_1=0}{P_{00}\land P_{10}} \\ (7)
	}
	{
		\hoare{P_{00}\land P_{10}}{done_0=0;done_1=0;(S_0 || S_1)}{Q_0\land Q_1}
	}
}
{
	\hoare{X=A\land Y=B}{x:=X;Y:=1 || y:=Y;X:=x}{x=A\lor y=B}
}
\end{equation*}
Transition from first to second line is rule of composition. Transition from second to third line is auxiliary variable rule that gets rid of $done_0$ and $done_1$ and the result is the wanted specification.
\end{proof}